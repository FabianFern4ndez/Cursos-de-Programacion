\documentclass[]{article}
\usepackage{geometry}
%Establecer margen superior
\geometry{top = 0.5cm}

%opening
\title{Tarea de Matemática Discreta}
\author{Fabián Fernández E-8-196729   \,\,\,Grupo:1IL122}
\date{26 de Marzo del 2024}

\begin{document}

\maketitle

\section*{Problemas Propuestos}
Demostrar ley de Morgan usando los conjuntos A y B \\ \\
Sea U = \{x : 1 $\leq$ x $\leq$ 10\} \\ \\
A = \{1,3,6,7,9,10\} \\ \\
B = \{1,2,3,7,9,10\} \\ \\
Recordar que \((A\cup B\))' = A' $\cap$ B'

\section*{Simbologia}
Pertenencia: x $\in$ Y , x $\notin$ Y \\ \\
Subconjunto: X $\subset$ Y , Y $\not\subset$ X \\ \\
Desigualdad: x $\leq$ y , y $\geq$ x \\ \\
Operaciones: X $\cup$ Y, X $\cap$ Y

\section*{Desarrollo}
Paso 1. Calcular A $\cup$ B \\ 
A $\cup$ B = \{1,2,3,6,7,9,10\} \\ \\
Paso 2. Calcular su complemento relativo a U\\
\((A \cup B\))' = \{4,5,8\} \\ \\
Paso 3. Calcular complementos de A y B \\
A' = \{2,4,5,8\} \, B' = \{4,5,6,8\}\\ \\
Paso 4. Calcular intersección de los complementos \\
A' $\cap$ B' = \{4,5,8\} \\ 

\begin{flushright}
Q.E.D.
\end{flushright}


\end{document}




