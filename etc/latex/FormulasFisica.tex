\documentclass[14pt]{extarticle}
\usepackage{geometry}

%Establecer margen superior
\geometry{top = 0.5cm}

%opening
\title{Tarea de Matemática Discreta}
\author{Fabián Fernández E-8-196729   \,\,\,Grupo:1IL122}
\date{26 de Marzo del 2024}

\begin{document}
    \section*{}

	\section*{\centering Cinemática}

	\section*{\\ \normalsize Movimiento Rectilíneo Uniforme}
	$\Delta \vec{x} = x_f - x_i$ \\\\
	$\Delta \vec{x} = \vec{v} t$ \\\\
 $\vec{v}_{instantanea} = \frac{dx}{dt}$
	\section*{\normalsize Movimiento Rectilíneo Uniforme Acelerado}
	$\Delta \vec{x} = \vec{v_i}t + \frac{1}{2} \vec{a}t^2$ \\\\
	$\vec{v_f} = \vec{v_i} + \vec{a}t$ \\\\
	$\vec{v_f}^2 = \vec{v_i}^2 + 2\vec{a}\Delta \vec{x}$\\\\
    $\vec{a}_{instantanea} =  \frac{dv}{dt}$
	\section*{\normalsize Caída Libre}
	Reemplazar $\vec{a}$ por $\vec{g} = -9,8 \frac{m}{s^2}$ , y $\Delta$x por $\Delta$y  en ecuaciones de MRUA \\\\
	Condición de Altura máxima $y_f$\\
	$\vec{v_{fy}} = 0$
	\section*{\normalsize Tiro Parabólico}
	$\vec{v}_{i} (vector) = \vec{v}_i cos(\theta) \hat{x} + \vec{v}_i sen(\theta) \hat{y} $  \,\,\,\,\,o  \,\,\,\,\, $\vec{v}_x = \vec{v}_icos(\theta)$\,,\,$\vec{v}_y = \vec{v}_isen(\theta)$ \\\\
	$\Delta \vec{x} = \vec{v}_x t$ \,\,\,\,MRU \\\\
	$\Delta \vec{y} = \vec{v}_yt + \frac{1}{2}\vec{g}t^2$ \,\,\,\,Caída Libre \\\\
	Ángulo: $\theta = arctan(\frac{\hat{y}}{\hat{x}})$ \,\,\,\,\,\,
	Magnitud: $\Delta = \sqrt{\hat{x}^2 + \hat{y}^2}$
	
	
	

	
\end{document}