\documentclass[14pt]{extarticle}
\usepackage{geometry}

%Establecer margen superior
\geometry{top = 0.5cm}

%opening
\title{Tarea de Matemática Discreta}
\author{Fabián Fernández E-8-196729   \,\,\,Grupo:1IL122}
\date{26 de Marzo del 2024}

\begin{document}
	\section*{}
	
	\section*{\centering Medidas de Tendencia Central}
	
	\section*{\\ \normalsize Datos no agrupados }
	Promedio o Media: Suma de datos/cantidad.\\\\
	Mediana: Ordenar datos y seleccionar el del medio, si la cantidad es par, hacer media de los dos centros.\\\\
	Moda: Valor que más se repite.
	
	\section*{\normalsize Datos Agrupados Continuos}
	Promedio = $\frac{\sum_{i}^{k} C_i n_i}{n}$ \\\\
	Mediana = $a_i + \frac{n/2 - N_{i-1}}{n_i} \times A$   \\ tomando como referencia(i) el intervalo donde esté el primer $N_i > n/2$\\\\
	Moda = $a_i + \frac{n_i - n_{i-1}}{(n_i - n_{i-1}) + (n_i - n_{i+1})} \times A$\\tomando como referencia(i) el intervalo con mayor $n_i$
	

	\section*{\normalsize Datos Agrupados Discretos}
	Promedio = $\frac{\sum_{i}^{n} x_i n_i}{n}$ \\\\
	Mediana si n/2 se encuentra en $N_i$ \\
$M_e = \frac{x_i + x_{i+1}}{2}$ , tomando como referencia(i) el $N_i = n/2$ \\\\
Mediana si n/2 NO se encuentra en $N_i$ \\
$M_e$ será el primer valor de $x_i$ con $N_i > n/2$ \\\\
Moda: categoría ($x_i$) con mayor $n_i$ \\\\\\\\\\\

	\section*{\normalsize Cuartiles}
	\section*{\normalsize Datos no Agrupados}
	N:Numero total de datos; K: Cuartil (1,2,3); $Q_k$:Posicion del cuartil K\\
	Si N es par $Q_k = K(\frac{N}{4})$\\
	Si N es impar $Q_k = K(\frac{N+1}{4})$\\
	Si el cuartil queda entre dos posiciones $Q_k = X_i + d \times (X_{i+1} - X_{i})$
	
	\section*{\normalsize Datos Agrupados}
	Paso 1: Calcular Posicion del cuartil $Q_k$\\
	Paso 2: Calcular el valor del cuartil
	$Q_k = a_i + \frac{k(\frac{N}{4}) - N_{i-1}}{n_i} \times A$ \\
	Tomando como referencia ai, es el intervalo inferior del primer $N_i > n/2$
	\section*{\normalsize Deciles y Percentiles}
	Se mantienen las ecuaciones de cuartiles, pero en vez de 4 ahora es 10 y 100 respectivamente\\\\
	 $D_k = K(\frac{N}{10})$\\
	 	$D_k = a_i + \frac{k(\frac{N}{10}) - N_{i-1}}{n_i} \times A$\\
	 	$P_k = K(\frac{N}{100})$\\
	 	$P_k = a_i + \frac{k(\frac{N}{100}) - N_{i-1}}{n_i} \times A$
	
	
	

	
	
	
	
	
\end{document}