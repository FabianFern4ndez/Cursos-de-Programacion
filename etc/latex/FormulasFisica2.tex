\documentclass[14pt]{extarticle}
\usepackage{geometry}

%Establecer margen superior
\geometry{top = 0.5cm}

%opening
\title{Tarea de Matemática Discreta}
\author{Fabián Fernández E-8-196729   \,\,\,Grupo:1IL122}
\date{26 de Marzo del 2024}

\begin{document}
	\section*{}
	
	\section*{\centering Fórmulas Parcial 2}
	
	\section*{\\ \normalsize Leyes del Movimiento }
	Ley de Inercia: $\sum \vec{F} = 0 $\\
	Ley de Cambio: $\sum \vec{F} = m\vec{a}$ \\
	Ley de Reacción: $\vec{F}_{A->B} = - \vec{F}_{B->A}$
	\section*{ \normalsize Fuerzas }
	1- Peso = m$\vec{g}$, se dibuja siempre hacia abajo(signo negativo) \\
	2- Normal(N): se dibuja perpendicular a la superficie (positiva, no hay fórmula, hallar mediante el peso)\\
	3- Fuerza de Empuje: Fuerza que causa el movimiento\\
	4- Fricción ($F_r$) = $N \mu_k$, se dibuja contraria al movimiento \\
	5- Tensión (T): Tensión
	\section*{\normalsize Trabajo y Energía }
	Trabajo(W) para F constante: $W = F \times \Delta x \times cos(\theta)$\\
	Trabajo Neto = $\sum W$\\
	Trabajo(W) para F variable: $W = \int_{x_1}^{x_2} F_x dx$\\
	Fuerza de Resorte = kx\\
	Trabajo del Resorte $W_{resorte} =\int kx dx = \frac{1}{2} k x_f ^2 -\frac{1}{2} k x_i ^2 $ \\\\
	Cambio de Energía Cinética ($\Delta E_c$) : $\Delta E_c = \frac{1}{2} m \vec{v_f} ^2 -\frac{1}{2} m \vec{v_i} ^2 $ \\
	Teorema: \ $W_{neto} = \Delta E_c $\\ 
	Potencia Media: $P_{med} = \frac{\Delta W}{\Delta t}$\\
	Potencia Instantánea: $P_{inst} = \frac{dW }{dt}$\\
	

\end{document}